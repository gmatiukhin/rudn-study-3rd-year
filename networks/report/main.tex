\documentclass[a4page]{article}
\usepackage[14pt]{extsizes}
\usepackage[utf8]{inputenc}
\usepackage[utf8]{inputenc}
\usepackage[russian]{babel}
\usepackage{amsmath}
\usepackage[left=20mm, top=15mm, right=15mm, bottom=30mm, footskip=15mm]{geometry}
\usepackage{indentfirst}
\usepackage{paralist}

\usepackage{fancyvrb}
\usepackage{framed}
\usepackage{url}
\usepackage{csquotes}

\usepackage{graphicx}
\usepackage{svg}
\graphicspath{ {./images/} }

\usepackage{float}
\floatstyle{ruled}

\usepackage[
	backend=biber,
	sorting=nyt,
	bibstyle=gost-numeric,
	citestyle=gost-numeric
]{biblatex}

\usepackage[
	bookmarks=true, colorlinks=true, unicode=true,
	urlcolor=black,linkcolor=black, anchorcolor=black,
	citecolor=black, menucolor=black, filecolor=black,
]{hyperref}

\addbibresource{sources.bib}

\renewcommand{\baselinestretch}{1.35}

\begin{document}


\begin{titlepage}

	\begin{center}
		\hfill \break
		\textbf{
			\large{РОССИЙСКИЙ УНИВЕРСИТЕТ ДРУЖБЫ НАРОДОВ}\\
			\normalsize{Факультет физико-математических и естественных наук}\\
			\normalsize{Кафедра прикладной информатики и теории вероятностей}\\
		}
		\vspace*{\fill}
		\Large{\textbf{ДОКЛАД\\ на тему <<Интранет и виртуальные частные сети>>}}
		\\
		\underline{\textit{\normalsize{Дисциплина: Сетевые технологии}}}
		\vspace*{\fill}

	\end{center}

	\begin{flushright}
		Студент: \underline{Матюхин Григорий}\\ \vspace{0.5cm}
		Группа: \underline{НПИбд-01-21}
	\end{flushright}


	\begin{center} \textbf{МОСКВА} \\ 2023 г. \end{center}
	\thispagestyle{empty} % выключаем отображение номера для этой страницы

\end{titlepage}

\newpage

\tableofcontents

\newpage
\section{Введение}
Цифровая связь имеет важное значение на современном рабочем месте. Поскольку организации обеспокоены безопасностью и производительностью, использование внутренней сети или интрасети в качестве дополнения или замены Интернета является жизнеспособным вариантом для многих предприятий.

Интранет всей организации может стать важным центром внутреннего общения и сотрудничества, а также обеспечить единую отправную точку для доступа к внутренним и внешним ресурсам. В своей простейшей форме интранет создается с использованием технологий локальных сетей (LAN) и кампусных сетей (CAN).\cite{rfc4364} Многие современные интрасети имеют в своей инфраструктуре поисковые системы, профили пользователей, блоги, мобильные приложения с уведомлениями\\ и планированием мероприятий.

\section{Интранет}
Интранет (англ. Intranet, также употребляется термин интрасеть) - в отличие от Интернета, это внутренняя частная сеть, принадлежащая, как правило, частному лицу, организации или крупному государственному ведомству.\cite{enwiki:intranet} Интранет похож на <<Интернет в миниатюре>>, который построен, как правило, на использовании протокола IP для обмена и совместного использования некоторой части информации внутри этой организации. Это могут быть списки сотрудников, списки телефонов партнёров и заказчиков. Чаще всего в разговорной речи под этим термином имеют в виду только видимую часть интранета - внутренний веб-сайт организации.\\ Основанный на базовых протоколах HTTP и HTTPS и организованный по принципу клиент-сервер, интранет-сайт доступен с любого компьютера через браузер. Таким образом, интранет - это <<частный>> Интернет, ограниченный виртуальным\\ пространством отдельно взятой организации. 

\subsection{Применение}
Интранет все чаще используется для доставки инструментов, например. совместная работа (для облегчения работы в группах и проведения телеконференций) или сложные корпоративные каталоги, инструменты управления продажами и взаимо\-отношениями с клиентами, управление проектами и т. д.,

Интранеты также используются в качестве платформ для изменения корпоративной культуры. Например, большое количество сотрудников, обсуждающих ключевые вопросы в приложении форума во внутренней сети, может привести к появлению новых идей в области управления, производительности, качества и других корпо\-ративных вопросов.

В крупных интрасетях трафик веб-сайтов часто аналогичен трафику общедоступных веб-сайтов, и его можно лучше понять с помощью программного обеспечения веб-метрики для отслеживания общей активности. Опросы пользователей также повышают эффективность веб-сайтов интрасети.

Группы по работе с пользователями интрасети, редакционные и технологические команды работают вместе над созданием собственных сайтов. Чаще всего интранетами управляют отделы коммуникаций, HR или CIO крупных организаций или их комбинация.

\subsection{Преимущества}
\begin{itemize}
\item Производительность труда\\
Интранет может помочь пользователям быстрее находить и просматривать информацию, а также использовать приложения, соответствующие их ролям и обязанностям. С помощью интерфейса веб-браузера пользователи могут получить доступ к данным, хранящимся в любой базе данных, которую организация хочет сделать доступной, в любое время и - при соблюдении мер безопасности - из любого места на рабочих станциях компании, что повышает способность сотрудников выполнять свою работу быстрее и эффективнее. точно и с уверенностью, что они владеют правильной информацией. Это также помогает улучшить услуги, предоставляемые пользователям.

\item Экономия времени\\
Интранет позволяет организациям распространять информацию среди сотрудников по мере необходимости. Сотрудники могут ссылаться на соответствующую информацию, когда им удобно, вместо того, чтобы отвлекаться на электронную почту без разбора.

\item Улучшение коммуникаций\\
Интранет может служить мощным инструментом коммуникации внутри\\ организации, вертикальными стратегическими инициативами, имеющими\\ глобальный охват по всей организации. Тип информации, которую можно легко передать, - это цель инициативы и то, чего она стремится достичь, кто руководит инициативой, достигнутые на сегодняшний день результаты и к кому можно обратиться за дополнительной информацией. Предоставляя эту информацию в интранете, сотрудники имеют возможность быть в курсе стратегических целей организации. Некоторыми примерами общения могут быть чат, электронная почта и/или блоги.

\item Расширение сотрудничества\\
Информация легко доступна всем авторизованным пользователям, что позволяет работать в команде. Возможность общаться в режиме реального времени с помощью интегрированных сторонних инструментов, таких как программы обмена мгновенными сообщениями, способствует обмену идеями и устраняет препятствия для общения, помогая повысить производительность бизнеса.

\item Деловые операции и управление\\
Интранет также используется в качестве платформы для разработки и\\ развертывания приложений для поддержки бизнес-операций и принятия решений в рамках объединенного предприятия.
\end{itemize}

\subsection{Недостатки}
Такое устройство интранета может работать только в пределах одной локации. Однако многие огранизации имеют сотрудников, работающих удаленно. Многие компании также имеют несколько офисов, потенциально находящихся в разных городах или странах. Сотрудникам этих офисов необходимо иметь доступ ко всем внутренним данным организации. Кроме того, многие организации сотрудничают друг с другом по тем или иным вопросам и при этом происходит огромный обмен информацией. Для таких целей получения доступа к части интранета партнера или предоставление доступа к части своего является хорошей идеей.

Решением такой задачи является создание виртуальной частной сети, соединяющей сотрудников, оффисы и партнеров вместе.

\section{Виртуальная частная сеть}
VPN (англ. Virtual Private Network - <<виртуальная частная сеть>>) - обобщённое название технологий, позволяющих обеспечить одно или несколько сетевых соединений поверх чьей-либо другой сети\cite{enwiki:vpn}\cite{nist-glossary:vpn}. Несмотря на то, что для коммуникации используются сети с меньшим или неизвестным уровнем доверия (например, публичные сети), уровень доверия к построенной логической сети не зависит от уровня доверия к базовым сетям благодаря использованию средств криптографии (шифрования, аутентификации, инфраструктуры открытых ключей, средств защиты от повторов и изменений передаваемых по логической сети сообщений).

Виртуальные частные сети, предоставляемые поставщиком услуг (Provider Provisioned Virtual Private Networks, PPVPNs), - это сети VPN корпоративного уровня, которые в основном используются предприятиями для обеспечения сотрудникам безопасного удаленного доступа к их корпоративной сети. PPVPN также используются для безопасного соединения физически отдельных сайтов и сетей друг с другом через Интернет.

\subsection{Категоризация}
Виртуальные частные сети можно разделить на несколько категорий:
\begin{itemize}
\item Удаленный доступ\\
Конфигурация хост-сеть аналогична подключению компьютера к локальной сети. Этот тип обеспечивает доступ к корпоративной интрасети. Это может быть использовано для удаленных работников или для предоставления мобильному работнику доступа к необходимым инструментам, не предоставля доступа через общедоступный Интернет.

\item Сеть-сеть\\
Конфигурация <<сеть-сеть>> соединяет две сети. Эта конфигурация расширяет сеть на географически разнесенные офисы или подключает группу офисов к центру обработки данных. Соединяющее соединение может проходить через разную промежуточную сеть, например две сети IPv6, соединенные через сеть IPv4.

\begin{figure}
\includesvg[width=500px]{./images/Virtual_Private_Network_Intranet_overview.svg}
\caption{VPN-соединение, показывающий совместное использование конфигураций типа <<сеть-сеть>> и удаленного доступа в интрасети.}
\end{figure}

\item Сеть-сеть для экстрасети\\
  В контексте конфигураций типа <<сеть-сеть>> термины <<интрасеть>> и <<экстранет>>\cite{enwiki:extranet} используются для описания двух разных вариантов использования.\cite{rfc3809} VPN типа <<сеть-сеть>> в интрасети описывает конфигурацию, в которой сайты, подключенные с помощью VPN, принадлежат одной и той же организации, тогда как VPN типа <<сеть-сеть>> в экстрасети объединяет сайты, принадлежащие нескольким организациям.
\end{itemize}

Обычно частные лица взаимодействуют с VPN удаленного доступа, тогда как предприятия, как правило, используют соединения типа <<сеть-сеть>> для сценариев <<бизнес-бизнес>>, облачных вычислений и филиалов. Однако эти технологии не являются взаимоисключающими и в значительно сложной бизнес-сети могут быть объединены для обеспечения удаленного доступа к ресурсам, расположенным в любом заданном месте, например, к системе заказов, расположенной в центре обработки данных.

\subsection{Устройство PPVPN}

\begin{figure}
\includesvg[width=500px]{./images/Site-to-Site_VPN_terminology-en.svg}
\caption{Терминология VPN типа <<сеть-сеть>>}
\end{figure}

\begin{itemize}
  \item Устройство клиента (C)\\
Устройство, находящееся в сети клиента и не подключенное напрямую к сети поставщика услуг. Устройства C не знают о VPN.

\item Пограничиное устройство клиента (CE)\\
Устройство на границе сети клиента, обеспечивающее доступ к PPVPN. Иногда это просто точка разграничения между ответственностью поставщика и клиента. Другие провайдеры позволяют клиентам настраивать его.

\item Пограничное устройство провайдера (PE)\\
Устройство или набор устройств на границе сети провайдера, которое подключается к сетям клиентов через устройства CE и представляет представление провайдера о сайте клиента. PE знают о VPN, которые подключаются через них,\\ и поддерживают состояние VPN.

\item Устройство провайдера (P)\\
Устройство, которое работает внутри базовой сети поставщика и не имеет прямого интерфейса с какой-либо конечной точкой клиента. Например, он может обеспечивать маршрутизацию для множества туннелей, управляемых провайдером, которые принадлежат PPVPN разных клиентов. Хотя устройство P является ключевым элементом реализации PPVPN, оно само по себе не поддерживает VPN и не поддерживает состояние VPN. Его основная роль заключается в том, чтобы позволить поставщику услуг масштабировать свои предложения PPVPN, например, выступая в качестве точки агрегации для нескольких PE. Соединения P-to-P в такой роли часто представляют собой оптические каналы высокой пропускной способности между основными\\ местоположениями провайдеров.
\end{itemize}


\newpage

\section{Список литературы}

% Print bibliography without heading
\printbibliography [heading=none]

\end{document}
