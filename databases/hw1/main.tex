\documentclass[12pt]{article}
\usepackage[T1, T2A]{fontenc}
\usepackage[utf8]{inputenc}
\usepackage[russian]{babel} 
\usepackage{datetime}

\usepackage{amssymb}
\usepackage{amsmath}

\usepackage{amsfonts}
\usepackage{ifsym}
 
\newcommand{\employee}{\textbf{СЛ}}
\newcommand{\department}{\textbf{ОТД}}
\newcommand{\depplace}{\textbf{О\_М}}
\newcommand{\project}{\textbf{ПР}}
\newcommand{\dependant}{\textbf{ИЖД}}
\newcommand{\workson}{\textbf{Р\_Н}}
 
\newcommand{\employeeName}{\textbf{СЛ}_\text{Имя}}
\newcommand{\employeeFirstname}{\textbf{СЛ}_\text{Имя}}
\newcommand{\employeeLastname}{\textbf{СЛ}_\text{Фамилия}}
\newcommand{\employeePatroname}{\textbf{СЛ}_\text{Отчество}}
\newcommand{\employeeId}{\textbf{СЛ}_\text{Код}}
\newcommand{\employeeGender}{\textbf{СЛ}_\text{Пол}}
\newcommand{\employeeAddress}{\textbf{СЛ}_\text{Адрес}}
\newcommand{\employeeSalary}{\textbf{СЛ}_\text{Зарплата}}
\newcommand{\employeeBday}{\textbf{СЛ}_\text{Дрожд}}
\newcommand{\employeeDepId}{\textbf{СЛ}_\text{Но}}
\newcommand{\employeeCuratorId}{\textbf{СЛ}_\text{КодК}}
 
\newcommand{\departmentName}{\textbf{ОТД}_\text{Назв}}
\newcommand{\departmentId}{\textbf{ОТД}_\text{Но}}
\newcommand{\departmentBossId}{\textbf{ОТД}_\text{КодР}}
 
\newcommand{\depplaceDep}{\textbf{О\_М}_\text{Но}}
\newcommand{\depplacePlace}{\textbf{О\_М}_\text{Место}}
 
 
\newcommand{\projectName}{\textbf{ПР}_\text{Назв}}
\newcommand{\projectId}{\textbf{ПР}_\text{Nп}}
\newcommand{\projectPlace}{\textbf{ПР}_\text{Место}}
\newcommand{\projectResponsibleDepId}{\textbf{ПР}_\text{Nоо}}
 
\newcommand{\dependantEmpId}{\textbf{ИЖД}_\text{Код}}
\newcommand{\dependantName}{\textbf{ИЖД}_\text{Имя}}
\newcommand{\dependantGender}{\textbf{ИЖД}_\text{Пол}}
\newcommand{\dependantBday}{\textbf{ИЖД}_\text{Дрожд}}
\newcommand{\dependantRelation}{\textbf{ИЖД}_\text{СтРод}}
 
\newcommand{\worksonEmpId}{\textbf{Р\_Н}_\text{Код}}
\newcommand{\worksonProjId}{\textbf{Р\_Н}_\text{Nп}}
\newcommand{\worksonTime}{\textbf{Р\_Н}_\text{Время}}

\newcommand{\filter}[2]{\sigma{}_{#1} \allowbreak \left( #2 \right)}
\newcommand{\join}[3]{ #1 \bowtie #2;\allowbreak #3 }
\newcommand{\select}[2]{\pi_{#1} \allowbreak \left( #2 \right)}
\newcommand{\aggregate}[4]{  #1 \mathbb{F}_{#2} \left[AS\ #3 \right] \left( #4 \right)  }

\usepackage{minted}

% hack for russian letters inside minted: https://tex.stackexchange.com/a/343506/293327
\makeatletter
\AtBeginEnvironment{minted}{\dontdofcolorbox}
\def\dontdofcolorbox{\renewcommand\fcolorbox[4][]{##4}}
\makeatother

\usepackage{tabularx}

\author{Григорий Матюхин}
\date{\formatdate{8}{10}{2023}}
\title{Реляционные базы данных. ДЗ}

\begin{document}
\maketitle
\newpage

\section{1-е занятие}
\subsection{6}

Имена служащих, имеющих, по крайней мере, одного иждивенца. \\\\
$ DependantsAndEmps = \join{\dependant}{\employee}{\dependantEmpId = \employeeId} $ 
  для иждивенца, его служащий; служащие без иждивенцев не включены  \\ 
$ Output = \select{\employeeName}{DependantsAndEmps} $ 
 вывести имена служащих \\ 

\subsection{7}
Имена служащих, работающих в отделе с названием «Информатика» над проектом с названием «Прогресс»\\\\
$ Dep = \filter{\departmentName = \text{<<Информатика>>}}{\department}$ 
 найти отдел \\ 
$ Project = \filter{\projectName = \text{<<Прогресс>>}}{\project}$ 
   найти проект \\ 
$ DepEmployee = \join{Dep}{\employee}{\departmentId = \employeeDepId}$ 
  все сотрудники в отделе <<Информатика>>  \\ 
$ DepEmployeeWorksOn = \join{DepEmployee}{\workson}{\employeeId = \worksonEmpId}$ 
 с информацией о сотруднике   \\ 
$ DepEmployeeWorksOnProject = \join{DepEmployeeWorksOn}{\project}{\worksonProjId = \projectId}$ 
  с информацией о проекте  \\ 
$ DepEmployeeWorksOnSpecificProject = \join{DepEmployeeWorksOnProject}{Project}{DepEmployeeWorksOnProject\rightarrow \projectId = Project\rightarrow\projectId}$ 
 этот проект должен быть проектом <<Прогресс>> \\ 
$ Output = \select{\employeeName}{DepEmployeeWorksOnSpecificProject}$ 
 вывести имена сотрудников из отдела с этим проектом   \\ 

\subsection{8}
Имена служащих и имена его иждивенцев.\\\\
$ DependantsAndEmps = \join{\dependant}{\employee}{\dependantEmpId = \employeeId} $ 
  для каждого иждивенца, имя его сотрудника  \\ 
$ Output = \filter{\employeeName; \dependantName}{DependantsAndEmps}$ 
  вывести имена обоих  \\ 

\subsection{9}
Названия проектов, за которые отвечает отдел, руководимый Ивановым.\\\\
$ Boss = \filter{\employeeLastname = \text{<<Иванов>>}}{\employee}$ 
  сотрудник Иванов  \\ 
$ Deps = \join{Boss}{\department}{\employeeId = \departmentBossId}$ 
  отделы, для которых он начальник  \\ 
$ Projects = \join{Deps}{\project}{\departmentId = \projectResponsibleDepId}$ 
   проекты, для которых он ответственный отдел \\ 
$ Output = \select{\projectName}{Projects}$ 
  вывести имена проектов  \\ 

\subsection{10}

Названия отделов, отвечающих за проекты с названием «Принцип …»\\\\
$ Projects = \filter{\projectName = \text{<<Принцип …>>}}{\project} $ 
  проекты с названием  \\ 
$ ProjectsAndDeps = \join{Projects}{\department}{\projectResponsibleDepId = \departmentId}$ 
 для проектов ответственный отдел   \\ 
$ Output = \select{\departmentName}{ProjectsAndDeps} $ 
   только имя отдела \\ 

\subsection{11}

Имена курируемых, работающих в том же отделе что и куратор.\\\\
$ Trainee = \employee$ 
 курируемый сотрудник   \\ 
$ Trainer = \employee$ 
 куратор сотрудник   \\ 
$ TraineesAndTrainers = \join{Trainee}{Trainer}{Trainee\rightarrow\employeeCuratorId = Trainer\rightarrow\employeeId}$ 
 второй сотрудник является куратором для первого  \\ 
$ SameDepTTs = \filter{Trainee\rightarrow\employeeDepId = Trainer\rightarrow\employeeDepId}{TraineesAndTrainers}$ 
  проверить, что у двух сотрудников один и тот же отдел  \\ 
$ Output = \select{Trainee\rightarrow\employeeName}{SameDepTTs}$ 
   вывести имя курируемого \\ 

\subsection{12}

Названия проектов, выполняемых в месте, в котором размещается отдел, отвечающий за проект.\\\\
$ ProjectsAndDeps = \join{\project}{\department}{\projectResponsibleDepId = \departmentId}$ 
    \\ 
$ ProjectsAndDepsAndDepPlaces = \join{ProjectsAndDeps}{\depplace}{\departmentId = \depplaceDep}$ 
    \\ 
$ ProjectsSamePlace = \filter{\projectPlace = \depplacePlace}{ProjectsAndDepsAndDepPlaces}$ 
    \\ 
$ Output = \select{\projectName}{ProjectsSamePlace}$ 
  вывести имя проекта  \\ 

\subsection{13}

Номера проектов, над которыми работает Иванов, но отдел, в котором он работает, не отвечает за проект.\\\\
$ Person = \filter{\employeeLastname = \text{<<Иванов>>}}{\employee} $ 
 только Иванов \\ 
$ PersonWorkson = \join{Person}{\workson}{\employeeId = \worksonEmpId}$ 
 над какими проектами работает Иванов? \\ 
$ PersonWorksonProject = \join{PersonWorkson}{\project}{\worksonProjId = \projectId}$ 
 полная информация об этом проекте \\ 
$ ProjectRespNotMine = \filter{\employeeDepId \neq \projectResponsibleDepId}{PersonWorksonProject}$ 
 отдел проекта -- не мой отдел \\ 
$ Output = \select{\projectId}{ProjectRespNotMine}$ 
 вывести только номер \\ 



\section{2-е занятие}
\subsection{2.3}
Выдать номера проектов, над которыми работают {только / все / только и все} сотрудники отдела, отвечающего за проект.
1. Выдать номера проектов, над которыми работают только сотрудники отдела, отвечающего за проект.
2. Выдать номера проектов, над которыми работают все сотрудники отдела, отвечающего за проект.
3. Выдать номера проектов, над которыми работают только и все сотрудники отдела, отвечающего за проект.\\\\
    $ EmpWorks = \join{\employee}{\workson}{\employeeId = \worksonEmpId}$ 
 сотрудники и номера их проектов \\ 
    $ EWP = \join{EmpWorks}{\project}{\worksonProjId = \projectId}$ 
 сотрудники и их проекты \\ 
    $ EP = \select{\employeeId; \projectId}{EWP}$ 
 номера всех сотрудников и их проектов \\ 
    $ EWPSelf = \filter{\employeeDepId = \projectResponsibleDepId}{EWP}$ 
 те работники, которые работают над проектом своего отдела \\ 
    $ EPS = \select{\employeeId; \projectId}{EWPSelf}$ 
 только номера сотрудников и проектов из одного отдела \\ 
    $ EPNotSelf = EP \setminus EPS$ 
 те сотрудники, которые работают над проектами не своего отдела, и эти проекты \\  
    $ PSelf = \select{\projectId}{EPS}$ 
 проекты, над которыми работает хотя бы кто-то из своего отдела \\ 
    $ PNotSelf = \select{\projectId}{EPNotSelf}$ 
 проекты, над которыми работает хотя бы кто-то не из своего отдела \\ 
    $ Output_1 = PSelf \setminus PNotSelf $ 
 проекты, над которыми работают люди из своего отдела, и не из какого другого \\ 
    $ Combos = \select{\employeeId; \employeeDepId}{EWP} \times \select{\projectResponsibleDepId; \projectId}{EWP} $ 
 все сочетания: (сотрудник и его отдел) и (проект и его отдел) \\ 
    $ CmbMatch = \filter{\employeeDepId = \projectResponsibleDepId}{Combos}$ 
 для каждого проекта, только люди которые работают в его отделе (несмотря на то, работают ли они над этим проектом) \\ 
    $ CMS = \select{\employeeId; \projectId}{CmbMatch}$ 
 все комбинации проектов и сотрудников, которые из одного отдела \\ 
    $ MissFull = CMS \setminus EPS$ 
 только те сотрудники, которые НЕ работают над проектами своего отдела, и проекты над которыми они НЕ работают \\ 
    $ MissFullProj = \select{\projectId}{MissFull}$ 
 те проекты, над которыми кто-то из их отдела не работает \\ 
    $ Output_2 = \filter{\projectId}{\project} \setminus MissFullProj$ 
 только те проекты, над которыми работают все из их отдела \\ 
    $ Output_3 = Output_1 \cap Output_2 $ 
 проекты, над которыми работают все сотрудники их отдела, и никого другого \\ 
% \aggregate{by column}{max(column)}{MaxColumnValue}{from table}
% min, max, sum, count, avg

\section{3-е занятие}
\subsection{5}
Названия отделов, численность которых больше 10 человек, а средняя зарплата меньше 500 рублей.\\\\
    $ DepPopulation = \aggregate{\employeeDepId}{count(1)}{Population}{\employee}$ 
 номер отдела и его население \\ 
    $ DPFilter = \filter{Population > 10}{DepPopulation}$ 
 отделы с населением больше 10 \\ 
    $ DepAvgSal = \aggregate{\employeeDepId}{avg(\employeeSalary)}{AvgSal}{\employee}$ 
 номер отдела и его средняя зарплата \\ 
    $ DASFilter = \filter{AvgSal < 500}{DepAvgSal} $ 
 отделы с средней зарплатой меньше 500 \\ 
    $ DepsOk = \join{DPFilter}{DASFilter}{\employeeDepId = \employeeDepId}$ 
 пересечение двух условий \\ 
    $ DepsOkData = \join{DepsOk}{\department}{\employeeDepId = \departmentId}$ 
 добавить информацию \\  
    $ Output = \select{\departmentName}{DepsOkData}$ 
 только имя \\ 

\subsection{6}

Номера проектов, для которых из всех работающих над ними более половины сотрудники отвечающего отдела.\\\\
    $ProjWorkers = \aggregate{\worksonProjId}{count(1)}{AllWorkers}{\workson}$ 
 для каждого проекта, число работников \\ 
    $ EmpWorksOn = \join{\employee}{\workson}{\employeeId = \worksonEmpId}$ 
 сотрудники и номера проектов \\ 
    $ EWOProj = \join{EmpWorksOn}{\project}{\worksonProjId = \projectId}$ 
 сотрудники и их проекты \\ 
    $ EWOPFilter = \filter{\employeeDepId = \projectResponsibleDepId}{EWOProj}$ 
 сотрудники и проекты их отдела \\ 
    $ ProjWorkersFromSelf = \aggregate{\projectId}{count(1)}{SelfWorkers}{EWOPFilter}$ 
 количество работников из отдела проекта \\ 
    $ ProjAWAndSW = \join{ProjWorkers}{ProjWorkersFromSelf}{\worksonProjId = \projectId}$ 
 номер проекта, кол-во работников всего, и кол-во работников отдела \\ 
    $ ProjSelfMajority = \filter{\frac{SelfWorkers}{AllWorkers} > 0.5}{ProjAWAndSW} $ 
 только проекты где своих работников большинство \\ 
    $ Output = \select{\projectId}{ProjSelfMajority}$ 
 вывести номера \\ 

\subsection{7}

Названия отделов, в которых более трех служащих имеют двух иждивенцев, родившихся в 2012 году.\\\\
    $ AgeDeps = \filter{\dependantBday LIKE 2012-??-??}{\dependant}$ 
 все иждивенцы в 2012 году \\ 
    $ EmpDepCounts = \aggregate{\dependantEmpId}{count(1)}{DepCount}{AgeDeps}$ 
 сотрудники и кол-во ижд. из 2012 \\ 
    $ EDCFilter = \filter{DepCount = 2}{EmpDepCounts}$ 
 сотрудники с ровно двумя ижд. из 2012 \\ 
    $ EDCFEmp = \join{EDCFilter}{\employee}{\dependantEmpId = \employeeId}$ 
 добавить инфу о сотруднике \\ 
    $ EmpsByDep = \aggregate{\employeeDepId}{count(1)}{EmpCount}{EDCFEmp}$ 
 посчитать удолетворяющих сотрудников по отделам \\ 
    $ DepsWithMany = \filter{EmpCount>3}{EmpsByDep}$ 
 отделы с больше 3 удолетворяющих сотрудников \\ 
    $ DWMDep = \join{DepsWithMany}{\department}{\employeeDepId = \departmentId}$ 
 добавить инфу об отделе \\ 
    $ Output = \select{\departmentName}{DWMDep}$ 
 вывести названия \\ 

\subsection{8}
Номера проектов, в работе над которыми наибольшее время затрачивает сотрудник отвечающего отдела.\\\\
    $ MPT = \aggregate{\worksonProjId}{max(\worksonTime)}{MaxWorkTime}{\workson}$ 
 какое самое длинное время работают над проектом? \\ 
    $ MaxWorker = \join{MPT}{\workson}{MPT\rightarrow\worksonProjId = \worksonProjId \wedge  MPT\rightarrow\worksonTime = \worksonTime}$ 
 кто работает такое время? \\ 
    $ MWProj = \join{MaxWorker}{\project}{\worksonProjId = \projectId}$ 
 добавить инфу про проект \\ 
    $ MWPEmp = \join{MWProj}{\employee}{\worksonEmpId = \employeeId}$ 
 добавить инфу про сотрудника, который работает самое длинное время над этим проектом \\  
    $ MWPESelf = \filter{\employeeDepId = \projectResponsibleDepId}{MWPEmp}$ 
 только те сотрудники, которые из отдела этого проекта \\ 
    $ Output = \select{\projectId}{MWPESelf}$ 
 вывести номера \\ 

\subsection{9}
Номера проектов, в работе над которыми наибольшее время затрачива[ю]т сотрудник[и] отвечающего отдела.\\\\
    $ EW = \join{\employee}{\workson}{\employeeId = \worksonEmpId}$ 
 полные сотрудники и проекты, над которыми они работают \\ 
    $ PDH = \aggregate{\left(\employeeDepId; \worksonProjId\right)}{sum(\worksonTime)}{TotHours}{EW}$ 
 для каждого проекта и отдела, сколько в сумме работают над проектом сотрудники этого отдела? \\ 
    $ PDMH = \aggregate{\worksonProjId}{max(TotHours)}{MaxTotHours}{PDH}$ 
 макс. время, которое отдел проработал над проектом \\ 
    $ PDMHDep = \join{PDMH}{PDH}{PDMH\rightarrow\worksonProjId = PDH\rightarrow\worksonProjId \wedge PDMH\rightarrow MaxTotHours = PDH\rightarrow TotHours}$ 
 какой отдел проработал это макс. время? \\ 
    $ MaxTimeDepProj = \join{PDMHDep}{\project}{PDMHDep\rightarrow\worksonProjId = \projectId}$ 
 добавить инфу о проекте \\ 
    $ MTDPFilter = \filter{\projectResponsibleDepId = \employeeDepId}{MaxTimeDepProj}$ 
 оставить только те отделы, которые отвечают за проект, над которым работали больше остальных отделов \\ 
    $ Output = \select{\projectId}{MTDPFilter}$ 
 вывести номера \\ 

\section{4-е занятие}
\subsection{2}

Номера проектов, имя руководителя отвечающего отдела и количество служащих, работающих над проектом при условии, что руководитель работает над не менее чем 2-мя другими проектами.\\\\
    $ ProjDeps = \join{\project}{\department}{\projectResponsibleDepId=\departmentId}$ 
 проект и его отдел \\ 
    $ PDB = \join{ProjDeps}{\employee}{\departmentBossId = \employeeId}$ 
 проект, отдел и его начальник \\ 
    $ ProjEmpCounts = \aggregate{\worksonProjId}{count(1)}{EmpCount}{\workson}$ 
 для каждого проекта, сколько работают над ним \\ 
    $ PDBC = \join{PDB}{ProjEmpCounts}{\projectId = \worksonProjId}$ 
 проект, отдел, начальник и количество служащих у проекта \\ 
    $ UnfilteredWithCode = \select{\projectId; \employeeId; \employeeName; EmpCount}{PDBC}$ 
 то, что нужно вывести, плюс код начальника, но без применения условия \\ 
    $ EmpProjCounts = \aggregate{\worksonEmpId}{count(1)}{BossProjCount}{\workson}$ 
 для каждого сотрудника, над сколькими проектами он работает \\ 
    $ EPCBig = \filter{BossProjCount \geq 3}{EmpProjCounts}$ 
 только те сотрудники, которые работают над одним и ещё двумя проектами \\ 
    $ UfAndEPC = \join{UnfilteredWithCode}{EPCBig}{UnfilteredWithCode\rightarrow\employeeId = EPCBig\rightarrow\worksonEmpId}$ 
 только такие проекты, где начальник работает над больше 3 проекта \\ 
    $ UEPCBossCheck = \join{UfAndEPC}{\workson}{\employeeId = \worksonEmpId; \projectId = \worksonProjId}$ 
 проекты, где начальник работает над >3 проектов, и один из них -- это данный проект \\ 
    $ Output = \select{\projectId; \employeeName; EmpCount}{UEPCBossCheck}$ 
 оставить только имя начальника в выводе \\ 

\end{document}
