\documentclass[a4page]{article}
\usepackage[12pt]{extsizes}
\usepackage[T2A]{fontenc}
\usepackage[utf8]{inputenc}
\usepackage{hyphsubst}
\usepackage[english,greek,russian]{babel}
\usepackage{amsmath}
\usepackage[left=20mm, top=15mm, right=15mm, bottom=30mm, footskip=15mm]{geometry}
\usepackage{indentfirst}
\usepackage{paralist}

\usepackage{fancyvrb}
\usepackage{framed}
\usepackage{url}
\usepackage{csquotes}

\usepackage{graphicx}
\usepackage{svg}
\graphicspath{ {./images/} }

\usepackage{float}
\floatstyle{ruled}

\usepackage[
	backend=biber,
	sorting=nyt,
	bibstyle=gost-numeric,
	citestyle=gost-numeric
]{biblatex}

\usepackage[
	bookmarks=true, colorlinks=true, unicode=true,
	urlcolor=black,linkcolor=black, anchorcolor=black,
	citecolor=black, menucolor=black, filecolor=black,
]{hyperref}

\addbibresource{sources.bib}

\renewcommand{\baselinestretch}{1.35}

\begin{document}


\begin{titlepage}

	\begin{center}
		\hfill \break
		\textbf{
			\large{РОССИЙСКИЙ УНИВЕРСИТЕТ ДРУЖБЫ НАРОДОВ}\\
			\normalsize{Факультет физико-математических и естественных наук}\\
			\normalsize{Кафедра математического моделирования и искусственного интеллекта}\\
		}
		\vspace*{\fill}
		\Large{\textbf{ДОКЛАД\\ на тему <<Методы кибернетики для описания социальных систем>>}}
		\\
		\underline{\textit{\normalsize{Дисциплина: Основы разработки корпоративных и инфокоммуникационных систем}}}
		\vspace*{\fill}

	\end{center}

	\begin{flushright}
		Студент: \underline{Матюхин Григорий}\\ \vspace{0.5cm}
		Группа: \underline{НПИбд-01-21}
	\end{flushright}


	\begin{center} \textbf{МОСКВА} \\ 2023 г. \end{center}
	\thispagestyle{empty} % выключаем отображение номера для этой страницы

\end{titlepage}

\newpage

\tableofcontents

\newpage

\section{Introduction}
The concept of sociocybernetics has been shaped over the past 40 years at the intersection between first- and second-order cybernetics, constructive epistemology and systems science. This has produced a common language to bridge these different disciplines and a common basis for research analysing complex social problems. Sociocybernetics applies second-order cybernetics concepts to the study of societies, communities and groups in which first- and second-order reflexivity may play an important part, but it is not only a theoretical perspective in the abstract; it is also an approach that is applied to the analysis of cross-disciplinary issues such as systemic violence, the role of technology in society, environmental challenges, urban planning, community development, social identity and media representation, among many others. In this report we will try to examine what sociocybernetics actually is and take a look at some of the methodology used by the scientists.

\newpage
\section{Кибернетика}
\selectlanguage{russian}
Кибернетика (от др.-греч. \selectlanguage{greek}κυβερνητική\selectlanguage{russian} <<искусство управления>>) — наука об общих закономерностях получения, хранения, преобразования и передачи информации в сложных управляющих системах, будь то машины, живые организмы или общество\cite{ruwiki:cybernetics}.

Кибернетика включает изучение обратной связи, чёрных ящиков и производных концептов, таких как управление и коммуникация в живых организмах, машинах и организациях, включая самоорганизации. Она фокусирует внимание на том, как что-либо (цифровое, механическое или биологическое) обрабатывает информацию, реагирует на неё и изменяется или может быть изменено, для того чтобы лучше выполнять первые две задачи\cite{out_of_control}.

Кибернетические методы применяются при исследовании случая, когда действие системы в окружающей среде вызывает некоторое изменение в окружающей среде, а это изменение проявляется на системе через обратную связь, что вызывает изменения в способе поведения системы. В исследовании этих <<петель обратной связи>> и заключаются методы кибернетики.
% TODO: fill out first order cybernetics

\subsection{Кибернетика второго порядка}
Кибернетика второго порядка, также известная как кибернетика кибернетики, является рекурсивным приложением кибернетики к самой себе.

Хейнц фон Фёрстер в статье <<Cybernetics of Cybernetics>> 1974 года, провёл различие между кибернетикой первого порядка — кибернетикой наблюдаемых систем, и кибернетикой второго порядка — кибернетикой наблюдающих систем\cite{cybernetics_of_cybernetics}.

Концепция кибернетики второго порядка тесно связана с радикальным конструктивизмом. Хотя иногда это считается отходом от более ранних проблем кибернетики, существует большая преемственность с предыдущими работами, и ее можно рассматривать как отдельную традицию в кибернетике, истоки которой очевидны во время конференций Мэйси, на которых кибернетика изначально разрабатывалась. Его проблемы включают автономию, эпистемологию, этику, язык, рефлексивность, самосогласованность, самореференциальность и возможности самоорганизации сложных систем.

Ключевая концепция кибернетики второго порядка заключается в том, что наблюдателей (и других действующих лиц, таких как дизайнеры, моделисты, пользователи...) следует понимать как участников систем, с которыми они взаимодействуют, в отличие от отстраненности, подразумеваемой объективностью. и традиционная научная практика. Это включает в себя включение самих кибернетиков в практику кибернетики, а также вовлечение участников в рассмотрение и проектирование систем в более общем плане.

Отношения кибернетики первого и второго порядка можно сравнить с взглядами Исаака Ньютона на Вселенную и взглядами Альберта Эйнштейна\cite{cybernetics_of_cybernetics}. Как описание Ньютона остается уместным и пригодным для использования во многих обстоятельствах, даже при полетах на Луну, так и кибернетика первого порядка дает все, что необходимо во многих обстоятельствах. Точно так же, как взгляд Ньютона понимается как особая, ограниченная версия взгляда Эйнштейна, кибернетика первого порядка может пониматься как особая, ограниченная версия кибернетики второго порядка.

Различие между кибернетикой первого и второго порядка иногда используется как форма периодизации. Однако это может затмить преемственность между ранней и поздней кибернетикой, поскольку так называемые качества второго порядка очевидны в работах таких кибернетиков, как Уоррен Маккалок и Грегори Бейтсон, а Ферстер и Мид являются одновременно участниками конференции Мэйси и ее инициаторами. кибернетики второго порядка. Мид и Бейтсон, например, отметили, что они с Винером считали себя участвующими наблюдателями в отличие от отстраненного подхода «ввод-вывод», типичного для инженерного дела. В этом смысле кибернетику второго порядка можно рассматривать как отдельную традицию внутри кибернетики, которая развивалась по направлениям, отличным от более узкого подхода к инженерной кибернетике.

\newpage
\section{Социальная кибернетика}
Социальная кибернетика (англ. Sociocybernetics) — независимый раздел в социологии, основанный на общей теории систем и кибернетике\cite{ruwiki:sociocybernetics}.

Основная цель создания социальной кибернетики — создание теоретической основы и инструментов информационной технологии для преодоления стандартных вызовов, с которыми на сегодняшний день сталкиваются отдельные личности, пары, семьи, компании, организации, страны и международные отношения.

\subsection{Теоретическая основа}
Социальная кибернетика имеет под собой цель создать общую теоретическую основу для понимания кооперативного поведения. Она утверждает, что предоставляет глубокое понимание общей теории эволюции. В ней говорится: Все живые системы проходят сквозь шесть уровней взаимоотношений (социальных ступеней) в своих подсистемах:

\begin{enumerate}
  \item Агрессия: выживи или умри
  \item Бюрократия: следуй нормам и правилам
  \item Соревнование: мои победы-твои поражения
  \item Договоренность: проявление личных чувств и целей
  \item Эмпатия: кооперация в рамках единого интереса
  \item Свободная воля: Способность любого вида существ вне зависимости от разновидности, расы, пола, вероучения, веры, генетики или сознания управлять своим существованием и не находиться под контролем.
\end{enumerate}

Прохождение сквозь эти шесть фаз взаимоотношений теоретически дает основу для социально кибернетического изучения эволюционной системы, в которой социальная кибернетика играет роль «уравнения жизни». Социальная кибернетика может быть определена как «Системная Наука в Социологии и других Социальных Науках» — системная наука потому что социальная кибернетика не ограничена теорией, а включает в себя так же и практическое применение, эмпирические исследования, методологию, аксиологию и эпистемологию. В общем употреблении «теория систем» и «кибернетика» зачастую взаимно заменяются или используются в комбинации. Таким образом они могут считаться синонимами хотя эти два определения имеют разные корни и не имеют полного распространения в других языках и национальных традициях.

Социальная кибернетика включает в себя как кибернетику первого, так и второго порядков. В междисциплинарном и холистическом духе, пусть социология и является центром внимания социальной кибернетики, остальные социальные науки такие как психология, антропология, политология, экономика тоже упоминаются с уклоном, зависящим от постановки научного вопроса. 

\newpage
% NOTE: this seems to be one of the methods that I need to be talking about
\section{Теория социальных систем Никласа Лумана}
Теория социальных систем Лумана фокусируется на трех темах, которые взаимосвязаны во всей его работе\cite{systemctheorie}.

\begin{itemize}
     \item Теория систем как теория общества
     \item Теория коммуникации
     \item Теория эволюции
\end{itemize}

Основной элемент теории Лумана вращается вокруг проблемы случайности значения и, таким образом, становится теорией коммуникации. Социальные системы — это системы коммуникации, а общество — это наиболее всеобъемлющая социальная система. Будучи социальной системой, которая включает в себя всю (и только) коммуникацию, сегодняшнее общество представляет собой мировое общество\cite{society_as_social_sytem}. Система определяется границей между собой и окружающей средой, отделяющей ее от бесконечно сложной или (в разговорной речи) хаотичной внешней среды. Таким образом, внутренняя часть системы представляет собой зону пониженной сложности: коммуникация внутри системы осуществляется путем выбора только ограниченного количества всей информации, доступной снаружи. Этот процесс еще называют «уменьшением сложности». Критерием отбора и обработки информации является смысл (по-немецки Sinn). Смысл здесь является переходом от одного набора потенциального пространства к другому набору потенциального пространства.

Более того, каждая система имеет свою отличительную идентичность, которая постоянно воспроизводится в ее коммуникации и зависит от того, что считается значимым, а что нет. Если системе не удается сохранить эту идентичность, она перестает существовать как система и растворяется обратно в среде, из которой возникла. Луман назвал этот процесс воспроизводства из элементов, ранее отфильтрованных из сверхсложной окружающей среды, аутопойезисом (буквально: самотворение), используя термин, придуманный в когнитивной биологии чилийскими мыслителями Умберто Матураной и Франциско. Варела. Социальные системы оперативно закрыты в том смысле, что, хотя они используют ресурсы своей среды и полагаются на них, эти ресурсы не становятся частью работы систем. И мышление, и пищеварение являются важными предпосылками общения, но ни то, ни другое не проявляется в общении как таковом\cite{society_as_social_sytem}.

Хотя Луман впервые развил свое понимание теории социальных систем под влиянием Толкотта Парсонса, вскоре он отошел от концепции Парсонса. Самое важное отличие состоит в том, что Парсонс рассматривал системы как формы действия в соответствии с парадигмой AGIL. Теория систем Парсонса рассматривает системы как оперативно открытые и интерактивные посредством схемы ввода и вывода. Под влиянием кибернетики второго порядка Луман вместо этого рассматривает системы как аутопойетические и оперативно закрытые\cite{social_systems}. Системы должны постоянно конструировать себя и свою точку зрения на реальность, осознавая различие между системой и окружающей средой, и самовоспроизводить себя как продукт своих собственных элементов. Социальные системы определяются Луманом не как действие, а как рекурсивная коммуникация. Современное общество определяется как мировая система, состоящая из суммы всех коммуникаций, происходящих одновременно, а отдельные функциональные системы (такие как экономика, политика, наука, любовь, искусство, средства массовой информации и т. д.) описываются как социальные подсистемы, которые «передифференцировались» от социальной системы и достигли собственной операционной замкнутости и аутопоэзиса\cite{theory_of_society_2}.

Еще одно отличие состоит в том, что Парсонс спрашивает, как определенные подсистемы способствуют функционированию общества в целом. Луман начинает с выделения самих систем из неописуемой среды. Хотя он действительно наблюдает, как определенные системы выполняют функции, вносящие вклад в «общество» в целом, он отказывается от предположения об априорном культурном или нормативном консенсусе или «дополняющей цели», которая была общей для концептуализации социальной функции Дюркгеймом и Парсонсом\cite{theory_of_society_1}. Для Лумана функциональная дифференциация является следствием селективного давления в условиях временной сложности, и она происходит, когда функциональные системы независимо устанавливают свои собственные экологические ниши, выполняя функцию. Таким образом, функции не являются скоординированными компонентами органического социального целого, а скорее случайными и избирательными ответами на референтные проблемы, которые не подчиняются никакому высшему принципу порядка и на которые можно было бы ответить другими способами.

Наконец, аутопойетическая замкнутость систем является еще одним фундаментальным отличием от концепции Парсонса. Каждая система работает строго по своему коду и может наблюдать за другими системами, только применяя свой код к их операциям. Например, кодекс экономики предполагает применение разграничения между оплатой и неплатежом. Другие системные операции появляются в экономическом поле отсчета лишь постольку, поскольку к ним может быть применен данный экономический кодекс. Следовательно, политическое решение становится экономической операцией, когда оно рассматривается как трата правительством денег или нет. Аналогичным образом, судебное решение также может быть экономической операцией, когда урегулирование договорного спора обязывает одну сторону оплатить приобретенные ею товары или услуги. Коды экономики, политики и права действуют автономно, но их «взаимопроникновение»\cite{social_systems} становится очевидным при наблюдении «событий»\cite{theory_of_society_2}, которые одновременно предполагают участие более чем одной системы.

\subsection{Mass media and the web in the light of Luhmann’s media system}
Еще в 1990 году Никлас Луман утверждал, что «если изменятся средства массовой информации и методы общения, если изменятся возможности и чувствительность выражения, если коды изменятся с устного общения на письменное и, прежде всего, если возможности воспроизведения и хранения увеличатся, новые структуры становятся возможными и, в конечном итоге, необходимыми, чтобы справиться с новыми сложностями"\cite{self_reference}.

Оглядываясь назад, слова Лумана кажутся пророческими. Высокий уровень сложности, достигнутый мировым обществом, то есть глобализованным обществом, включая все системы коммуникации, организации и взаимодействия, является неизбежной отправной точкой для социологического исследования, которое рассматривает коммуникацию и ее средства как элементы общества, его структура и процессы трансформации.

В этом смысле теоретический аппарат социокибернетики, понимаемой здесь как наука о социальных системах и конструктивистская парадигма, основанная на кибернетике второго порядка, дает ряд концепций и указаний, которые позволяют нам наблюдать это состояние сложности через саму сложность. Это означает описание общества и его функциональных систем как позиций, с которых можно наблюдать сложность, а не как контекстов, которые каким-то образом пытаются разрешить ее путем ее уменьшения.

С точки зрения социокибернетики, вопрос повышенной сложности может быть решен прежде всего через экологический подход. Эталонной моделью является Грегори Бейтсон\cite{steps_to_ecology_of_mind} и точная настройка эпистемологии, которая сложилась в области теории систем и кибернетики. С 1950-х годов стало очевидно, что понятие экологии, имеющее отношение к окружающей среде, стало абстрактным, обобщенным понятием до такой степени, что теперь мы говорим об общей экологии. Окружающая среда — это не данность, нечто существующее само по себе, к чему организм или, как в нашем случае, система должен адаптироваться, а «многогранный и гибкий ориентир, который меняется в зависимости от способа наблюдения и перспективы». наблюдателя"\cite{ecology_of_differences}. На этой основе экологический подход рассматривает систему как отдельную от окружающей ее среды, но при этом учитывает их взаимосвязи; следовательно, вместо того, чтобы стирать различия между системой и окружающей средой, наблюдение за ними умножает их. Поэтому, если, приняв точку зрения наблюдателя, мы будем относиться к средствам массовой информации как к системе, а к Интернету как к ее среде, то вопрос, который необходимо решить, касается условий этого различия и различных форм, которые оно принимает.

Наблюдать за средствами массовой информации сегодня — значит принимать во внимание то, как центральную роль средств массовой информации в контексте мирового общества следует рассматривать по отношению к коммуникативной среде, создаваемой Интернетом. Вот почему настоящий анализ берет за отправную точку определение средств массовой информации, предложенное Никласом Луманом в «Реальности средств массовой информации\cite{reality_of_mass_media}», классической работе о средствах массовой информации в свете социокибернетической парадигмы. В рамках теоретической основы общей теории социальных систем Луман выделяет средства массовой информации, с одной стороны, как специфическую функциональную систему, самореферентную и автопоэтическую, а с другой — различные средства распространения массовой коммуникации. . Как социальная система, в контексте структуры, основанной на функциональной дифференциации, система СМИ оперирует через информационный/неинформационный код и действует через свои конкретные организации (каналы вещания, издательства и т.п.): «Система может Информация, таким образом, представляет собой положительную ценность, обозначающую ценность, с помощью которой система описывает возможности своего собственного функционирования. Но для того, чтобы иметь свободу видеть что-то как информацию или нет, должна быть также возможность думать, что что-то неинформативно"\cite{reality_of_mass_media}. Функция системы средств массовой информации заключается в создании общей для всех второй реальности, базовых знаний, которые можно воспринимать как нечто само собой разумеющееся (например, если мы упомянем чемпионат мира по футболу 2018 года, все поймут, о чем мы говорим, даже если мы не можем этого сделать). следуй за ним). Эти помещения необходимы для того, чтобы производить новые коммуникации без необходимости начинать все заново. Конструирование реальности средствами массовой информации проистекает из наблюдения второго порядка, поскольку, наблюдая за средствами массовой информации, человек наблюдает то, что наблюдают другие. Тем не менее, это не реальность, которая требует консенсуса: она касается только того, что наблюдается, а не того, как); в этом смысле мнение пользователей остается свободным.

Система средств массовой информации структурно связана с другими функциональными системами (экономикой, наукой, искусством, политикой и т. д.) – концепция, которая в теории социальных систем объясняет, как две автономные системы могут быть взаимно связаны, сохраняя при этом свою взаимную автономию. Окружающая среда обладает способностью запускать внутренние процессы возмущением или раздражением. Если какие-то события или эффекты окружающей среды заставляют систему приспосабливаться к окружающей среде, это структурная связь, поскольку система не подвергается никакому непосредственному вмешательству, а просто действует внутри для корректировки или адаптации. С точки зрения производства коммуникаций, система средств массовой информации включает в себя средства распространения – прессу, радио, кино, телевидение – которые обычно снижают долгосрочный порог невероятности коммуникации. Средства массовой информации развиваются из письма и поэтому отделены от личного взаимодействия, типичного для межличностного общения. Более того, они управляют своими отношениями с окружающей средой – личностью, понимаемой как психическая система и органическая система – на основе своей собственной организационной структуры, которая зависит от внутренней дифференциации между тремя программными областями – новости и углубленные репортажи, развлечения и реклама – которые могут пересекаться, но имеют свои особенности и критерии обращения с информацией (и отличия ее от того, что не является информацией).

В этом смысле поведение индивидов и их мотивы не влияют на функционирование программ, а, проще говоря, «человек» задействован исключительно как социальный конструкт, как тема общения и представляемый субъект. Новости и подробные репортажи предполагают, что люди хотят быть информированными; реклама предполагает наличие человека, который ищет собственной выгоды и которому, будучи способным к принятию решения, должны быть предоставлены возможности для принятия решения; в развлечениях, которые, по мнению Лумана, основаны на повествовательном вымысле, индивидом является человеческое существо, морально ответственное, но склонное впадать в искушение, которое поэтому необходимо объяснить разницу между хорошим поведением и плохим поведением. В конечном счете, «конструкт «когнитивно более или менее информированного, компетентного, морально ответственного человека» помогает функциональной системе средств массовой информации постоянно раздражать себя по отношению к своему биологическому и психическому человеческому окружению»\cite{reality_of_mass_media}. С этой точки зрения, для Лумана аудитория — это «исключенная третья», фантазия внутри системы: «это паразитическая роль в том смысле, что это не позиция, с которой человек может выполнять что-либо в соответствии с действиями системы». Функциональная система».

Помимо программ, темы коммуникации необходимы для синхронизации средств массовой информации и общества без ущерба для оперативного закрытия системы. Внутри системы темы представляют собой гетерореференцию коммуникации: средства массовой информации говорят о такой теме, как здоровье, и таким образом вступают в связь с медицинской системой. Луман приводит пример СПИДа, чтобы показать, что эта тема «не является продуктом самих средств массовой информации». Они просто воспринимают его, а затем обращаются с ним определенным образом, подчиняя тематическую траекторию, которую нельзя объяснить ни медицинскими диагнозами, ни общением между врачами и пациентами»\cite{reality_of_mass_media}. Таким образом, темы гармонизируют гетерореференцию (отсылку к внешнему миру) и самореференцию (отсылку внутрь медиа, его функции) внутри собственной коммуникации системы.

Поэтому темы важны для структурной связи средств массовой информации с другими социальными системами. Таким образом, средства массовой информации могут охватить все сферы общества (Луман говорит об «универсализме их функций»), но другие системы изо всех сил пытаются навязать свои темы средствам массовой информации и обеспечить их адекватное освещение. Центральное место средств массовой информации в современном обществе обусловлено их способностью навязывать принятие тем и устанавливать повестку дня. Такое принятие не зависит от принятия положительной или отрицательной позиции по отношению к информации. Действительно, интерес к этой теме часто является результатом того, что она вызывает противоречивые, полярные реакции.

\section{Sociocybernetic methods}
% TODO: tiny intro

\subsection{Models in sociocybernetics}
Абстрактные изображения использовались со времен каменного века для демонстрации наблюдений за объектами реального мира, но настоящий прорыв в моделировании произошел вместе с культурами Древнего Ближнего Востока и Древней Греции. Числа были первыми узнаваемыми моделями. Их можно датировать 30 000 годом до нашей эры. Следующими областями, в которых модели сыграли роль, были астрономия и архитектура примерно в 4000 году до нашей эры, а к 2000 году до нашей эры вавилонская, египетская и индийская культуры использовали математические модели для улучшения своей повседневной жизни.

Модель – это человеческий способ справиться с реальностью. Построение моделей для решения реальных проблем настолько важно для человеческого развития, что аналогичные методы были независимо разработаны в Китае, Индии и Персии.

Хотя почти любое подразделение Вселенной можно наблюдать и рассматривать как систему, наблюдатели, моделирующие системы, обычно устанавливают границы, которые минимизируют взаимодействие между наблюдаемой системой и остальной частью Вселенной.

В тех немногих случаях, когда социокибернетики действительно проверяли свои теории эмпирически, они использовали две разные стратегии\cite{validation_of_sociocybernetic_models}: стратегию операционализации и модельную стратегию.

В первой стратегии социокибернетик извлекает из теории набор предсказаний, каждое из которых можно проверить статистическими тестами. Во второй стратегии проверяемая теория не разбивается на набор предсказаний, а полностью переводится в компьютерную симуляционную модель, а затем проверяется путем проверки, может ли модель воспроизвести те самые процессы, которые теория, как предполагается, объясняет.

Тогда социокибернетикам нужны модели для проверки своих теорий, задача кибернетики первого порядка, но они должны сосредоточиться на цели разработчика модели, а не на модели, чтобы иметь задачу второго порядка, и это сложная, а иногда и конфликтная ситуация.

Идей по включению социокибернетических идей в модель немного, и они в основном связаны с назначением модели. Четырехэтапный метод, предложенный ван дер Зувеном\cite{validation_of_sociocybernetic_models}:

\begin{enumerate}
  \item Представляем самоссылающийся элемент управления. Между переменными состояния и их входными и выходными переменными добавляются петли обратной связи.
   \item Учет возмущений со стороны окружающей среды. Очень маловероятно, чтобы система функционировала в полностью статичной среде; можно было бы, по крайней мере, ожидать, что произойдут некоторые случайные возмущения.
   \item Представляет целеустремленность. Поскольку основной концепцией кибернетики является управление, для кибернетических моделей важно, чтобы они отражали целенаправленное поведение.
   \item Учет морфогенеза. Предполагается, что социальные системы имеют возможность изменять свою структуру в процессе своего функционирования.
 \end{enumerate}

Предстоящая задача — лучше понять и перевести концепции социокибернетики в модели, которые можно будет протестировать с помощью компьютерного моделирования. Однако, как признал ван дер Зувен, основные концепции, такие как самоорганизация, самовоспроизведение или самореференция, трудно перевести в набор уравнений.

Следствием этого является необходимость создания нового моделирования и новых методологий тестирования моделей, чтобы предоставить социокибернетикам средства для адекватной проверки своих представлений о мире.

\subsection{Sociocybernetics and participatory action research}
Исследование совместных действий (Participatory action research, PAR) — это подход к исследованию действий, в котором особое внимание уделяется участию и действиям членов сообществ, затронутых этим исследованием. Он стремится понять мир, пытаясь изменить его, совместно и после размышлений. PAR делает упор на коллективные исследования и эксперименты, основанные на опыте и социальной истории. В рамках процесса PAR «сообщества исследований и действий развиваются и решают вопросы и проблемы, которые важны для тех, кто участвует в качестве соисследователей». PAR контрастирует с основными методами исследования, которые делают упор на контролируемое экспериментирование, статистический анализ и воспроизводимость результатов\cite{enwiki:par}.

PAR использовался как методология социального вмешательства, которая сочетает в себе исследовательские исследования и действия по анализу и вмешательству в сложные социальные проблемы.

Корни и развитие социокибернетики объединяют множество областей знаний. Социокибернетика решает проблемы применения анализа к сложным социальным проблемам, которые понимаются как ситуации и процессы, которые для оптимального понимания едва ли могут быть классифицированы по их соответствию конкретной дисциплине, поскольку концептуализация их элементов не может быть отделена или изучена изолированно. поскольку они являются взаимоопределенными\cite{sistemas_complejos}. Это проблема, с которой сталкивается большинство дисциплин и которая ведет к развитию строгих междисциплинарных исследовательских процессов. В этом разделе мы объясняем, насколько жизнеспособен системный взгляд на PAR и как концепции социокибернетики могут способствовать укреплению такой методологии.

Большая часть литературы, основанной на теории систем и кибернетике, согласна с тем, что система определяется произвольно, преследуя конкретную цель, которая определяет внутренние (элемент/компоненты, подсистемы, отношения) и внешние (контекстные) элементы. Тем не менее, существует согласие по некоторым важным характеристикам: все части системы должны присутствовать, чтобы система могла оптимально выполнять свое предназначение; части системы должны быть организованы определенным образом, чтобы она могла достичь своей цели; системы имеют конкретные цели внутри более крупных систем; системы сохраняют свою стабильность посредством колебаний и корректировок; а действия системы включают обратную связь, то есть способ, которым система снабжает себя информацией и регулирует свои собственные действия (Андерсон и Джонсон, 1997). Системы можно считать сложными, если они состоят из подсистем, элементы которых сильно разнородны и части которых не могут быть разделены до взаимозависимости с взаимной зависимостью их функций; у них есть процессы на разных уровнях, которые связаны друг с другом структурными отношениями, которые могут даже перекрываться; и они поддерживают плотность связей как внутри, так и за пределами системы, которая определяется операционным замыканием\cite{sistemas_complejos}.

Как утверждает Гарсиа, «в исходном пункте расследования не дается никакой системы. Система не определена, но она поддается определению. Правильное определение может возникнуть только в ходе самого расследования и для каждого случая в отдельности». Это включает в себя логику PAR, которая на первом этапе требует определения и разграничения системы, подлежащей изучению, с помощью подходов, основанных на участии.

Для лучшего понимания сложных социальных проблем и эффективного поиска их решения социокибернетика опирается на построение моделей: метод, связывающий обработку информации с вычислительной помощью и конструктивистскими структурами. Это происходит потому, что кибернетика второго порядка, как основа социокибернетики, унаследовала свою заботу о решении проблем от теории систем.

В то время как PAR делает упор на наблюдении за действиями, производимыми в процессе исследования, социокибернетика фокусируется на наблюдении за действиями и процессами внутри анализируемой системы – самого исследования – и, на следующем этапе, на размышлении, полученном в результате метанаблюдения за процессом наблюдения. Рассматривая PAR как систему наблюдения, социокибернетика обеспечивает наблюдение второго порядка, которое наблюдает за исследовательским процессом, составляющим саму систему наблюдения. В рамках PAR это требует интеграции регистрации и анализа всего процесса, что поможет сделать обучение системы более подотчетным.

Оба PAR подчеркивают важность участия участников в систематизации и анализе данных для достижения социальных преобразований, и, таким образом, частью цели исследовательской модели является повышение их организационного потенциала для генерации информации и знаний по вопросам, которые коллективно определены как актуальные.

PAR объединяет критический подход с участием в конкретном контексте сообщества/группы и подчеркивает важность анализа исторической и социальной структуры проблемы, подлежащей анализу.

PAR подчеркивает субъективность участников и ее влияние на процесс; социокибернетика использует концепцию рефлексивности наблюдателя и рефлексивности второго порядка для рассмотрения ситуации, в которой актор или участник наблюдает и размышляет о процессе, в котором участвует, или дальнейший наблюдатель наблюдает за таким участием, создавая его отражение. . В то время как PAR определяет сложность в соответствии с взаимоотношениями участников, социокибернетика рассматривает сложность с точки зрения уровней взаимодействия между подсистемами, из которых она состоит.

\newpage
\section{Conclusion}

The entire history of cybernetics and systems thinking is inextricably connected with the development of new technologies. It all began with a quantifiable measure of information, a simple communication system built around a sender, a receiver, a channel, and the theory of probability applied to address new issues, from improving transmissions to predicting the behaviour of particles and the trajectory of aircraft. Radar providing signals to guide ballistic tracking systems is used as a metaphor for studying feedback loops in humans and machines. The interdisciplinary efforts of scholars from different backgrounds, some united by the Macy Conferences, led to the proliferation of new ideas that influenced the development of artificial intelligence, computer science, public opinion polls, content analysis, human cognition and, of course, general theories about society. With this common background, approaching the study of society in terms of observing its systems, elements, networks of elements, boundaries and environments seems to work particularly well when it comes to understanding present times.

The more our digital interactions leave permanent traces, the easier it is to observe social systems. Ultimately, it also makes the behaviour of those social systems more unpredictable and harder to steer. The constant opportunity for real-time self-observation that the new digital technologies provide, enacts a process of reflexivity that can in turn lead to changes in our opinions and behaviours. As in classic feedback loops, these changes will in turn be observable and observed themselves, potentially leading to further change.

From this perspective, sociocybernetics is the science of turbulent societies that somehow continue to adapt themselves, despite the complexities they are confronted with. The contributions to this monograph present some of the current work of the research committee and showcase the extensive potential for sociocybernetics' interaction with other areas of sociology and contemporary debate.

\newpage
% \section{Список литературы}
% Print bibliography without heading
% \printbibliography [heading=none]
\addcontentsline{toc}{section}{Список литературы}
\printbibliography

\end{document}
